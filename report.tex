%----------------------------------------------------------------------------------------
%	PACKAGES AND DOCUMENT CONFIGURATIONS
%----------------------------------------------------------------------------------------

\documentclass{article}
% Use images in Latex
\usepackage{graphicx}
% "Figure", "Table", etc. translated to Dutch:
\usepackage[dutch]{babel}
% Use EC fonts for better compability with different kind of Operating Systems:
\usepackage[T1]{fontenc}
% Use Cite package to generate citates
\usepackage{cite}
% Enable text in equations
\usepackage{amsmath}
% Enable URL's
\usepackage{hyperref}
% Manipulate floating
\usepackage{float}
% Gantt chart
\usepackage{pgfgantt}
% Dutch spellcheck
% https://www.spelling.nu/
\usepackage{listings}
\lstset{language=C}
% code snippets

%----------------------------------------------------------------------------------------
%	DOCUMENT INFORMATION
%----------------------------------------------------------------------------------------

\title{OpenCL n-body}
\author{Vandevelde,~Simon~(\texttt{simon.vandevelde@student.kuleuven.be})
  \and
  Van~Assche,~Dylan~(\texttt{dylan.vanassche@student.kuleuven.be})}
\begin{document}
\maketitle %Create title

%----------------------------------------------------------------------------------------
%	SECTION 1: TEST ENVIRONMENT
%----------------------------------------------------------------------------------------
\section{Testomgeving}
\subsection{Compilatieproblemen}

Door het gebruik van verouderde packages zat er een bug in het \texttt{GLM} package wat
ervoor zorgde dat de code niet gecompileerd kon worden in onze testomgeving. Na het
bijwerken van \texttt{GLM} waren de problemen van de baan.

\subsection{Hardware}
\begin{itemize}
    \item CPU
    \item GPU
    \item RAM
    \item OS
\end{itemize}

%----------------------------------------------------------------------------------------
%	SECTION 2: CHANGES
%----------------------------------------------------------------------------------------
\section{OpenCL aanpassingen}
\subsection{Eerste for-lus}
De eerste for-lus, die we geparalleliseerd hebben, berekent de nieuwe snelheid van een lichaam.
Hiervoor moet de host de positie en snelheden data overbrengen naar de GPU, wachten tot de GPU
klaar is en dan de nieuwe snelheden ophalen van de GPU.

\begin{lstlisting}[caption={De eerste for-lus}, label={code:for1}, breaklines=true, basicstyle=\footnotesize]
for (int i = 0; i < length; ++i)
{
    for (int j = 0; j < length; ++j)
    {

        if (i == j)
            continue;

        cl_float3 pos_a = host_pos[i];
        cl_float3 pos_b = host_pos[j];

        float dist_x = (pos_a.s[0] - pos_b.s[0]) * distance_to_nearest_star;
        float dist_y = (pos_a.s[1] - pos_b.s[1]) * distance_to_nearest_star;
        float dist_z = (pos_a.s[2] - pos_b.s[2]) * distance_to_nearest_star;


        float distance = sqrt(
                dist_x * dist_x +
                dist_y * dist_y +
                dist_z * dist_z);

        float force_x = -mass_grav * dist_x / (distance * distance * distance);
        float force_y = -mass_grav * dist_y / (distance * distance * distance);
        float force_z = -mass_grav * dist_z / (distance * distance * distance);

        float acc_x = force_x / mass_of_sun;
        float acc_y = force_y / mass_of_sun;
        float acc_z = force_z / mass_of_sun;

        host_speed[i].s[0] += acc_x * delta_time;
        host_speed[i].s[1] += acc_y * delta_time;
        host_speed[i].s[2] += acc_z * delta_time;
    }

}
\end{lstlisting}

\subsection{Tweede for-lus}
De tweede for-lus, die we geparalleliseerd hebben, berekent de nieuwe positie
van elk lichaam. Hiervoor zal de host eerst de positie en snelheden overdragen
naar de GPU, wachten tot de GPU klaar is en dan de nieuwe positie data ophalen
van de GPU.

\begin{lstlisting}[caption={De tweede for-lus}, label={code:for2}, breaklines=true, basicstyle=\footnotesize]
for(int i = 0; i < length; ++i)
{
    host_pos[i].s[0] += (host_speed[i].s[0] * delta_time) / distance_to_nearest_star;
    host_pos[i].s[1] += (host_speed[i].s[1] * delta_time) / distance_to_nearest_star;
    host_pos[i].s[2] += (host_speed[i].s[2] * delta_time) / distance_to_nearest_star;
}
\end{lstlisting}

\subsection{Atomische variant van de eerste for-lus}
Omdat het algoritme in for-lus \ref{code:for1} de afstand bepaalt
tussen twee lichamen krijgen we een \textit{race conditie}. Het probleem situeert
zich bij het feit dat een GPU processor de positie van lichaam A en lichaam B moet
lezen om de afstand van lichaam B tot lichaam A te vinden. De kans bestaat dat een
andere GPU processor net de positie van lichaam A aan het wijzigen is terwijl deze
net ingelezen wordt. Het gevolg hiervan is dat de waarden verschillen per keer dat
we het programma uitvoeren.

We kunnen dit omzeilen door gebruik te maken van \textit{atomische operaties}.
Hierbij wacht een processor met het inlezen van een geheugenplaats als een andere
processor net naar deze geheugenplaats aan het schrijven is en omgekeerd. Een
nadeel van deze methode is dat de performantie zal dalen omdat de processor blijft
wachten tot een andere processor klaar is met zijn dataoverdracht.

\subsection{Combinatie van beide for-lussen}
\label{hfd:combinatie-lussen}
Hierbij hebben we de twee bovenstaande for-lussen (atomische for-lus \ref{code:for1} en for-lus \ref{code:for2}) gecombineerd tot \'{e}\'{e}n
OpenCL programma. Beide lussen staan nog steeds in een aparte kernel. Indien we
deze nog zouden combineren tot \'{e}\'{e}n kernel zouden we waarschijnlijk de
performantie nog kunnen verbeteren (zie \ref{hfd:dataoverdrachten}).

\subsection{Afstanden}
Een extra uitbreiding aan het programma was het opvragen van de afstanden tussen
elk lichaam en de andere lichamen:

\begin{itemize}
    \item Gemiddelde afstand
    \item Minimale afstand
    \item Maximale afstand
\end{itemize}



%----------------------------------------------------------------------------------------
%	SECTION 3: RESULTS
%----------------------------------------------------------------------------------------

\section{Resultaten}
Om onze resultaten te verwerken hebben we een klein Python script geschreven dat
alle timestamps inleest en het gemiddelde neemt van de 100 eerste metingen per
categorie. Deze gemiddelden worden dan uitgezet op een aantal grafieken welke
hieronder zullen worden besproken.

\subsection{5 lichamen}

\subsection{50 lichamen}

\subsection{500 lichamen}

\subsection{5 000 lichamen}

\subsection{50 000 lichamen}

\subsection{500 000 lichamen}
CPU faalt!

%----------------------------------------------------------------------------------------
%	SECTION 4: IMPROVEMENTS
%----------------------------------------------------------------------------------------
\section{Mogelijke verbeteringen}
Het is nog mogelijk om de laatste versie van het programma nog te versnellen door
gebruik te maken van nog enkele OpenCL functies en ontwerppatronen. Deze werden
niet ge\"{i}mplementeerd wegens tijdsgebrek.

\subsection{Minder dataoverdrachten}
\label{hfd:dataoverdrachten}
Het programma kopieert nu telkens per OpenCL kernel de data van de host naar de GPU en
omgekeerd. Dit resulteert in vier dataoverdrachten, wat erg veel nutteloze
overhead is, omdat beide for-lussen draaien als een aparte kernel.

We zouden de for-lussen kunnen overzetten in \'{e}\'{e}n kernel waardoor de dataoverdrachten
worden gereduceerd tot slechts twee overdrachten.

\subsection{Effici\"{e}nt opsplitsen in werkgroepen}
\label{hfd:werkgroepen}
Beide for-lussen worden simpel weg op de GPU uitgevoerd zonder dat het algoritme
geoptimaliseerd werd voor parrallelisme op de GPU. Dit resulteert in een slechter
gebruik van het geheugen (globaal i.p.v. lokaal) waardoor de berekeningen trager
kunnen verlopen door het geheugen bottleneck.

Elke processor kan aan de hele dataset. Beter gezegd: het geheugen is volledig
gedeeld onder elke processor. Hierdoor staat dit globaal geheugen verder weg
van de processor kern en duurt het dus langer vooraleer de data arriveert bij
de processor.

Als we gebruik maken van het lokaal geheugen van een werkgroep kunnen we dit probleem
voorkomen. Maar dit vereist tevens ook dat we de berekeningen zodanig kunnen opsplitsen
zodat ze in een werkgroep passen.

\subsection{Coalesced memory access}
Indien we het programma zouden optimaliseren om zijn geheugentoegangen te limiteren
tot zijn buren kunnen we een hogere snelheid halen omdat de naburige werkitems
sneller toegankelijk zijn dan werkitems die aan de andere kant van de GPU liggen.
Maar dit vereist opnieuw een ingreep op het algoritme zoals reeds besproken in
\ref{hfd:werkgroepen}, welke iets uitgebreider is omdat men rekening moet houden
met de buren van elk workitem.

%----------------------------------------------------------------------------------------
%	SECTION 5: CONCLUSION
%----------------------------------------------------------------------------------------
\section{Conclusie}

%----------------------------------------------------------------------------------------
\end{document}
